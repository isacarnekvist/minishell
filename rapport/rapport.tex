\documentclass{article}

\usepackage{verbatim}
\usepackage{fullpage}
\usepackage{hyperref}
\hypersetup{
    urlcolor=cyan,
    colorlinks=true,
}
\author{Isac Arnekvist 860202-2017 isacar@kth.se}
\title{miniShell and printenv}
\begin{document}
\pagenumbering{gobble}
\maketitle
\begin{verbatim}
                                   _       _ __ _          _ _ 
                         _ __ ___ (_)_ __ (_) _\ |__   ___| | |
                        | '_ ` _ \| | '_ \| \ \| '_ \ / _ \ | |
                        | | | | | | | | | | |\ \ | | |  __/ | |
                        |_| |_| |_|_|_| |_|_\__/_| |_|\___|_|_|
                                                               
\end{verbatim}
\section*{Summary}
    This describes the overall method of implementing a simple shell and a tool
    to list your environment variables. It took about 30-40 hours to complete
    and was made as two separate projects. It was a challenge to get everything
    to work cross platform with backwards compatibility. If done again: factor
    out prompt code from rest of code and try to do unit testing. Also, the use
    of pipes in the shell might be unnecessary.

\newpage
\pagenumbering{arabic}
\section{Introduction}
    I started this task right after the course started by reading the instructions
    on the \emph{ID2206} course webpage. I have not had any lab partner.
    
\section{Task}
    The task is to make a simple shell and a tool to list your environment
    variables. The shell should be able to handle foreground and background
    process and time the foreground ones. The tool for listing environment
    variables should take a regular expression as its argument to only show
    those lines matching it.  I initially followed the instructions from the
    course \emph{ID2206}, which turned out not being the same as this course.
    This led to \emph{miniShell} and \emph{digenv} being made as two separate
    executables. Of course, if \emph{digenv} is in your PATH, you can start it
    from anywhere using \emph{miniShell}. The same holds for changing to the
    \emph{digenv} folder and executing the local binary. For exact requirements
    see
    \href{https://www.kth.se/social/course/ID2200/page/laborations-pm/}{here}.

\section{Method}
    \subsection{Input}
        Input was done using the library \emph{readline}, which enables tab
        completion and also can enable history. The latter was never included
        since there was a bug when implemented and I felt that I did not want
        to spend time on it until everything else worked. After that, I check
        for \texttt{\&}, remove it and then tokenize the rest of the input.
        The reason not tokenizing before removing it is that in bash, it
        works both when separated with a whitespace character and when not,
        and the tokenizer I use, for simplicity, splits on whitespaces. So
        when the input arguments are tokenized, an interpret method is
        invoked to check for special cases like built-in commands like
        \emph{exit}. If not a special case, then forks are made and I try
        to execute the commands
        given. 
    \subsection{Polling}
        This was implemented in a little peculiar fashion, the reason is
        that I started with timing the background processes as well.
        The shell always forks of a child A, which in turn forks off an
        other child B. Child B then executes the arguments given. The
        reason for this is that child A can time B and wait for it no matter
        if the shell returns immediately to the prompt. If it is supposed to
        be run as a background process, the shell doesn't wait for process A
        to finish and returns to the prompt while A still can wait in the
        background with a timer running. When B finishes, A stopped the timer
        and sent the data over a pipe using that as a cue of terminated
        processes no matter if it was a background or foreground process.
        I had problems when doing the same with signal detection, so to
        have a congruent system, timing is no longer done of background
        processes. When the interpret function returns, the main loop has a
        function to poll the pipe for finished processes. The data sent over
        the pipe contains the process id and the time it was executing.
        Since A was never waited for, an additional wait is issued to clear
        it off the process table. This has a (not very pretty) downside. That
        is that when issuing the \emph{ps} command, both the shell and the A
        process shows up as ./minishell in the table. Sometimes the second one
        also gets the \texttt{<defunct>} tag because it had not cleared it yet
        before printing. This is \emph{not} the same as leaving a zombie
        process though, the process always get cleaned up right away.

    \subsection{Signal detection}
        To use signal detection for background processes instead, I do not use
        waitpid when \texttt{\&} was given. Instead, I return to the
        prompt and a signal handler is assigned to catch \texttt{SIGCHLD}
        signals. From the signal handler, I send all the processes that
        finished to the main loop via a pipe. The pipe is then read at an
        appropriate moment. There is a tricky problem here, who knows if foreground
        process terminated first, waitpid or the signal handler? If the signal handler
        acknowledges a terminated process before the waitpid, I will have an error.
        This is solved by using \texttt{sighold} and \texttt{sigrelse} while a
        foreground process is being run. This merely delays all signal handling until
        waitpid got to acknowledge the foreground process. Then when the signals
        are turned back on again, signal handler will automatically be invoked
        and a loop handles any non-acknowledged processes from the process table.
        The problem I had here was with timing, if I time it from the signal handler,
        the time might be way to long of the foreground process was running for a
        longer time than the background processes.
 
    \subsection{cd}
        Most of the commands you use work out of the box, the exception is
        \texttt{cd}. This has to change the current working directory for
        the process, so if a fork is made, this change would die with that
        process. So how solve this? I use the system call chdir and give
        the second argument as argument to this call. The downside of just
        giving the second argument is that if I would do the following:
        \begin{verbatim}
            cd .. foo
        \end{verbatim}
        This would lead to a change to the parent directory no matter if foo is
        an existing directory or not.

   \subsection{Ctrl-C}
        Since Ctrl-C should only started processes and not \emph{miniShell}
        itself, I simply assigned a handler for SIGINT which only returns.
        Something that is different from how bash handles this is if I start
        a couple of sleeps in the background, and then one in the foreground,
        all the sleeps gets killed by the Ctrl-C.

   \subsection{digenv}
        This was solved basically by forking four times in a row and setting up three
        pipes to communicate between them.
        \begin{verbatim}
            printenv | grep | sort | pager
        \end{verbatim}
        The communication is to redirect printenv's stdout to grep's stdin and so on.
        Some of the tricky parts was to close the pipes at the right moment. As an
        example, sort naturally waits for EOF before it can start sorting its input.
        So therefore, the pipe has to be closed before the pager pipe is closed.
        
\section{Verification}
    Only manual testing was done to make sure the \emph{miniShell} lives up to
    the requirements. Separate functions were tested after the introduction of each
    and one of them by doing manual testing, print statements and asserts.
    Regarding testing of the final shell, these following main themes were tested:
    \subsection{Simple usage}
        Basic existent and non-existent command was entered to make sure the
        behaviour is as expected. Some of the commands entered were:
        \begin{verbatim}
            > vim test.c
              ...
            > sleep 5
              ...
            > sleep 5 &
              ...
            > echo hej
              ...
            > ps
              ...
            > foo
              ...
        \end{verbatim}

    \subsection{Concurrent processes}
        I manually did:
        \begin{verbatim}
            > sleep 5 &
            > sleep 4 &
            > sleep 6
        \end{verbatim}
        with SIGSET set to both 0 and 1 to make sure that background processes
        that terminates while a foreground process is running behaves as
        expected. The expected result is that termination is not noticed until
        foreground process has finished. Also that all processes started give a
        notice about their termination. I also check with \emph{ps} so that no
        \emph{sleep} process is still there and that none remains as a
        \emph{defunct} process.

    \subsection{digenv}
        This was also tested manually by first leaving the environment variable
        PAGER unset and testing, then setting to less and more to see that it
        behaved as expected. Then I tried to give arguments to see that correct
        lines were filtered out. If a regex was given that did not match, it
        should of course not show any lines at all.

    \subsection{Possible tests to do}
        A lot of more automated tests could be done if the shell was to be used
        over a longer time and maintained. Although, I think som major effort
        is going to be needed to automatically check the forks and also check
        process tables for expected behaviour. Especially when the ''view'' and
        ''controller'' is in the same file.
        

\section{Installation and manual}
    Both the \emph{miniShell} and the \emph{digenv} comes with a compile script
    which compiles with flags \emph{gcc -pedantic -Wall -ansi -O4 ...} and also a
    run script which first compiles and then runs the compiled binary. (Not very
    needed for digenv, but for symmetry...) The files are in my home directory,
    isacar, under \texttt{minishell/}. The digenv folder also lies within minishell
    folder.

\section{Other}
    I needed maybe about 30-40 hours to complete the assignment. I think it was
    interesting, especially how hard it was to write code with these restrictions
    that does not give warnings and work on both macintosh and a linux server.
    If I started the project again I would have done some things different. That
    includes not having such a complicated fork process, since timing is not needed
    on background processes. I also would have split the program into a \emph{(M)VC}
    pattern to allow unit testing of all ''non-prompt'' functions.

    Now that I know that timing of background processes is not needed, I think
    that the use of pipes might no longer be needed. Prints could be done from
    the signal handler instead of sending data to the main function. On the
    other hand, I am not quite sure that prints from within the signal handler
    is guaranteed to work all the time. I have trouble finding this in the
    manuals, but I feel I have seen it somewhere. It might be something with
    that signal handlers are asynchronous and two calls might be called at
    approximately the same time. So what happens if another handler is called
    in the middle of a print within an other handler?

\newpage
\section{Appendix}
    \verbatiminput{../minishell.c}
    \verbatiminput{../digenv/digenv.c}
    \verbatiminput{../helpers.c}
\end{document}
